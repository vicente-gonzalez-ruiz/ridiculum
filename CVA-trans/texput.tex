% Curriculum Vitae Abreviado
% Modelo propuesto por el Ministerio de Economía y Competitividad

\documentclass[a4paper]{article}
\usepackage[utf8]{inputenc}
\usepackage[T1]{fontenc}
\usepackage[spanish,activeacute]{babel}
\usepackage[hidelinks]{hyperref}
\usepackage{eurosym}
\usepackage{color}

\textwidth=1.4\textwidth
\textheight=1.1\textheight
\hoffset=-2.5cm
\voffset=-1cm
\sloppy

%\newcommand{\clearemptydoublepage}{\newpage{\pagestyle{empty}\cleardoublepage}}

%\pagestyle{fancyplain} % Activa fancyplain

% Definición de las cabeceras
%\cfoot[]{} % Elimina el número a pie de página

\definecolor{dark-red}{rgb}{0.4,0.15,0.15}
\definecolor{dark-blue}{rgb}{0.15,0.15,0.4}
\definecolor{medium-blue}{rgb}{0,0,0.5}

\hypersetup{
  pdfauthor = {González Ruiz, Vicente},
  pdftitle = {CVA (transferencia)},
  pdfsubject = {I+D+i},
  %pdfkeywords = {Keyword1, Keyword2, ...},
  pdfcreator = {LaTeX with hyperref package},
  pdfproducer = {pdflatex},
  colorlinks,
  pdfborder={0 0 0},
  linkcolor={dark-red},
  citecolor={dark-blue}, 
  urlcolor={medium-blue},
  linktocpage
}

\title{Curriculum Vitae Abreviado (sólo transferencia)}
\author{
  \begin{tabular}{c}
    \input{nombre} \input{apellidos} \\\\
    \framebox{\href{http://www.hpca.ual.es/~vruiz}{Versión online}}
  \end{tabular}
}

\begin{document}
\def\normalfont{\sffamily}
\sffamily
%\pagestyle{empty} % Elimina las cabeceras
\maketitle

\section*{Datos personales}
\begin{tabular}{ll}
  Nombre y apellidos: & \input{nombre} \input{apellidos} \\
  DNI/NIE/pasaporte: & \input{DNI} \\
  Edad: & \input{edad} \\
  Researcher ID: & \input{ResearcherID} \\
  Código Orcid: & \input{ORCID}
\end{tabular}

\section*{Situación profesional actual}

\begin{tabular}{ll}
  Organismo: & \input{organismo} \\
  Dpto./Centro & \input{departamento} \\
  Dirección: & \input{direccion} \\
  Teléfono: & \input{telefono} \\
  Correo electrónico: & \input{email} \\
  Categoría profesional: & \input{categoria} \\
  Fecha inicio: & \input{inicio} \\
  Espec. cód. UNESCO: & \input{cods_UNESCO} \\
  Palabras clave: & \input{palabras_clave}
\end{tabular}

%\subsection{Formación académica}
%
%\begin{tabular}{lll}
%  Licenciatura/Grado/Doctorado & Universidad & Año \\
%  \hline
%  \input{doctor_en} & \input{doctorado_por} & \input{doctor_anho} \\
%  \input{licenciado_en} & \input{licenciado_por} & \input{licenciado_anho} \\
%  \input{diplomado_en} & \input{diplomado_por} & \input{diplomado_anho}
%\end{tabular}

% http://eventos.crue.org/_files/_event/_23313/_editorFiles/file/Documentos/Presentaciones/20181128_1500_1630_Cerezo_Lidia.pdf
\section*{Indicadores generales de la transferencia del conocimiento}
\begin{tabular}{ll}
Patentes: & 0 \\
Modelos de utilidad: & 0 \\
Organizaciones: & 1 (ver \hyperref[sec:organizations]{Organizaciones de software libre}) \\
Acuerdos de explotación: & 0 \\
Contratos con empresas: & 5 (ver \hyperref[sec:contratos]{Contratos})\\
Convenios I+D+i: & 1 (ver \hyperref[sec:convenios]{Convenios}) \\
Creación de empresas de base tecnológia: & 2 (ver \hyperref[sec:empresas]{Creación de empresas})\\
Proyectos colaborativos: & 8 (ver \hyperref[sec:proyectos]{Proyectos}) \\
Workshops (con empresas): & 1 (ver \hyperref[sec:workshops]{Workshops})
\end{tabular}

%\subsection{Indicadores generales de calidad de la producción científica}
%
%\begin{enumerate}
%\item Sexenios de investigación: \input{sexenios_anhos}
%\item Número de tesis dirigidas: \input{num_tesis_dirigidas}
%\item Artículos en revistas incluídas en el JCR: \input{num_articulos_JCR}
%\item Libros: \input{num_libros}
%\item Capítulos de libro: \input{num_capitulos_libro}
%\item \url{https://www.researcherid.com/rid/G-9269-2015}
:
%\begin{verbatim}
%Total Articles in Publication List:     20
%Articles With Citation Data:            19
%Sum of the Times Cited:                 76
%Average Citations per Article:          4.00
%h-index:                                4
%\end{verbatim}

%\item \url{https://scholar.google.es/citations?hl=es&user=RvBRplIAAAAJ
}
:
%\begin{verbatim}
%                Total   Desde 2014
%Citas           270     110
%Índice h        8       5
%Índice i10      6       1
%\end{verbatim}
%\end{enumerate}

%\section{Resumen libre del currículum}
%\noindent (Máximo 3500 caracteres, incluyendo espacios en blanco)
%\input{resumen_libre}

\renewcommand{\refname}{}

\section*{Organizaciones}
\label{sec:organizations}
\vspace{-5ex}
\documentclass{article}
\usepackage{hyperref}
\usepackage[utf8]{inputenc} 
\usepackage[T1]{fontenc}

\begin{document}
\documentclass{article}
\usepackage{hyperref}
\usepackage[utf8]{inputenc} 
\usepackage[T1]{fontenc}

\begin{document}
\documentclass{article}
\usepackage{hyperref}
\usepackage[utf8]{inputenc} 
\usepackage[T1]{fontenc}

\begin{document}
\input{../organizations}
\bibliography{vruiz} 
\bibliographystyle{organization}
\end{document}

\bibliography{vruiz} 
\bibliographystyle{organization}
\end{document}

\bibliography{vruiz} 
\bibliographystyle{organization}
\end{document}


\section*{Contratos}
\label{sec:contratos}
\vspace{-5ex}
\documentclass{article}
\usepackage{hyperref}
\usepackage[utf8]{inputenc} 
\usepackage[T1]{fontenc}

\begin{document}
% Contratos con empresas y otras entidades

\cite{JPIP-server}
\cite{FSVC_mov}
\cite{FSVC}
\cite{eSpectia}
%\cite{Jose} -> Pasado a convenios
\cite{TEDIAL-1}
%\cite{DICOM}

\bibliography{vruiz} 
\bibliographystyle{contrato}
\end{document}


\section*{Convenios}
\label{sec:convenios}
\vspace{-5ex}
\documentclass{article}
\usepackage{hyperref}
\usepackage[utf8]{inputenc} 
\usepackage[T1]{fontenc}

\begin{document}
% Convenios

\cite{Jose}

\bibliography{vruiz} 
\bibliographystyle{contrato}
\end{document}


\section*{Creación de empresas}
\label{sec:empresas}
\vspace{-5ex}
% Compañías creadas
\cite{Luxunda}
\cite{IPTV_solution}


\section*{Proyectos}
\label{sec:proyectos}
\vspace{-5ex}
\documentclass{article}
\usepackage{hyperref}
\usepackage[utf8]{inputenc} 
\usepackage[T1]{fontenc}
\usepackage{eurosym}

\begin{document}
\documentclass{article}
\usepackage{hyperref}
\usepackage[utf8]{inputenc} 
\usepackage[T1]{fontenc}
\usepackage{eurosym}

\begin{document}
\documentclass{article}
\usepackage{hyperref}
\usepackage[utf8]{inputenc} 
\usepackage[T1]{fontenc}
\usepackage{eurosym}

\begin{document}
\input{../proyectos_transferencia}
\bibliography{vruiz} 
\bibliographystyle{proyecto}
\end{document}

\bibliography{vruiz} 
\bibliographystyle{proyecto}
\end{document}

\bibliography{vruiz} 
\bibliographystyle{proyecto}
\end{document}


\section*{Workshops}
\label{sec:workshops}
\vspace{-5ex}
\documentclass{article}
\usepackage{hyperref}
\usepackage[utf8]{inputenc} 
\usepackage[T1]{fontenc}

\begin{document}
\documentclass{article}
\usepackage{hyperref}
\usepackage[utf8]{inputenc} 
\usepackage[T1]{fontenc}

\begin{document}
\documentclass{article}
\usepackage{hyperref}
\usepackage[utf8]{inputenc} 
\usepackage[T1]{fontenc}

\begin{document}
\input{../workshops}
\bibliography{vruiz} 
\bibliographystyle{workshop}
\end{document}

\bibliography{vruiz} 
\bibliographystyle{workshop}
\end{document}

\bibliography{vruiz} 
\bibliographystyle{workshop}
\end{document}


%\subsection{Patentes}

\end{document}
