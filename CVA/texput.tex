% Curriculum Vitae AbreviadoA
% Modelo propuesto por el Ministerio de Economía y Competitividad

\documentclass[a4paper]{article}
\usepackage[utf8]{inputenc}
\usepackage[T1]{fontenc}
\usepackage[spanish,activeacute]{babel}
\usepackage{hyperref}

\usepackage{catchfile}
\CatchFileDef\ResearcherID{ResearcherID.tex}{\endlinechar=-1}
\CatchFileDef\GoogleUser{Google_user.tex}{\endlinechar=-1}

\textwidth=1.4\textwidth
\textheight=1.1\textheight
\hoffset=-2.5cm
\voffset=-1cm
\sloppy

%\newcommand{\clearemptydoublepage}{\newpage{\pagestyle{empty}\cleardoublepage}}

%\pagestyle{fancyplain} % Activa fancyplain

% Definición de las cabeceras
%\cfoot[]{} % Elimina el número a pie de página

\title{Curriculum Vitae Abreviado}
\author{
  \begin{tabular}{ll}
    \textbf{Nombre:} & \input{nombre} \\
    \textbf{NIF:} & \input{apellidos}
  \end{tabular}
}

\begin{document}
\def\normalfont{\sffamily}
\sffamily
%\pagestyle{empty} % Elimina las cabeceras
\maketitle

\section{Datos personales}

\begin{tabular}{ll}
  Nombre y apellidos: & \input{nombre} \input{apellidos} \\
  DNI/NIE/pasaporte: & \input{DNI} \\
  Edad: & \input{edad} \\
  Researcher ID: & \input{ResearcherID} \\
  Código Orcid: & \input{ORCID}
\end{tabular}

\subsection{Situación profesional actual}

\begin{tabular}{ll}
  Organismo: & \input{organismo} \\
  Dpto./Centro & \input{departamento} \\
  Dirección: & \input{direccion} \\
  Teléfono: & \input{telefono} \\
  Correo electrónico: & \input{email} \\
  Categoría profesional: & \input{categoria} \\
  Fecha inicio: & \input{inicio} \\
  Espec. cód. UNESCO: & \input{cods_UNESCO} \\
  Palabras clave: & \input{palabras_clave}
\end{tabular}

\subsection{Formación académica}

\begin{tabular}{lll}
  Licenciatura/Grado/Doctorado & Universidad & Año \\
  \hline
  \input{doctor_en} & \input{doctorado_por} & \input{doctor_anho} \\
  \input{licenciado_en} & \input{licenciado_por} & \input{licenciado_anho} \\
  \input{diplomado_en} & \input{diplomado_por} & \input{diplomado_anho}
\end{tabular}

\subsection{Indicadores generales de calidad de la producción científica}

\begin{enumerate}
\item Sexenios de investigación: \input{sexenios_anhos}
\item Número de tesis dirigidas: \input{num_tesis_dirigidas}
\item Artículos en revistas incluídas en el JCR: \input{num_articulos_JCR}
\item Libros: \input{num_libros}
\item Capítulos de libro: \input{num_capitulos_libro}
\item \href{http://www.researcherid.com/rid/\ResearcherID}{Información en ResearcherID}.
\item \href{https://scholar.google.es/citations?hl=es&user=\GoogleUser}{Información en Google Académico}.
\end{enumerate}

\section{Resumen libre del currículum}
%\noindent (Máximo 3500 caracteres, incluyendo espacios en blanco)
\input{resumen_libre}
  
\section{Méritos más relevantes}
%\noindent (De los últimos 10 años)

\subsection{Publicaciones}
\input{top_5_publicaciones}

\subsection{Proyectos I+D+i}
%\noindent (Máximo 7)
\input{top_7_proyectos}

\subsection{Contratos I+D+i}
%\noindent (Máximo 7)
\input{top_7_contratos}

%\subsection{Patentes}

\subsection{Dirección de trabajos}
\cite{rodriguez2018estimacion}
\cite{OpenStack}
\cite{P2PSP-Chromecast}
\cite{Lowlatency}
\cite{P2PSP_WebRTC}
\cite{cliente_JPIP_android}
\cite{servidor_MCLTW_escalable_tiempo}
\cite{cliente_MCLTW_escalable_tiempo}
\cite{eSpectia_HTML5}
\cite{MCTF_plus_MJ2K}
\cite{MP3-scalable}
\cite{SCAMP}
\cite{Estimacion_mov_SVTE}
\cite{TranscoderMPEG}
\cite{MPEG4_multicast}
\cite{VideoProgresivo}
\cite{TransProgJPEG2000}
\cite{VisualizadorAstronomico}
\cite{VisualizadorRemoto}
\cite{Alineamiento}
\cite{VisSDLC++}


\subsection{Participación en tareas de evaluación}
\input{evaluador}

\subsection{Revisor de revistas}
\input{revisor}

\end{document}
